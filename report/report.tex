%TODO remove draft
\documentclass[11pt,a4paper,draft]{article}

\title{Genetic Algorithms}
\author{Alexander D Brown (adb9)}

\begin{document}
\maketitle
\tableofcontents

\newpage
\section{Introduction}
Genetic algorithms are a biologically-inspired approach to heuristic search 
which mimic natural selection. Unlike many other evolutionary strategies and
evolutionary programming, they are not designed to solve a specific problem,
but are designed to solve the problem of optimisation which is made difficult
by substantial complexity and uncertainty\cite{Holland1992Adaptation}.

The complexity of the task should make it such that discovering an optimum
solution is a long, maybe even impossible, task. At the same time the 
uncertainty needs to be reduced so that the knowledge of \textit{available}
options can be increased.

% TODO improve this section as its mainly from the reference.
The initial design for a genetic algorithm was a method for moving from one
population of chromosomes to another using a form of natural selection. This
algorithm also included methods for crossover, mutation and inversion. This
idea of having a large population was the distinguishing feature from any past
attempts which had only considered the parent and one offspring, where the 
offspring was simply a mutation of the parent\cite{Mitchell1996Introduction}.

\newpage
\bibliographystyle{plain}
\bibliography{citations}
\end{document}
